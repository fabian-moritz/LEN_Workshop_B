\section{Aufgabe 2}
Nun wird eine Addiererschaltung entsprechend einer vorgegebenen Übertragungsfunktion aufgebaut. Anschließend wird experimentiert, welchen Einfluss die verschiedenen Bauteilwerte innerhalb des Addierers und des invertierenden Verstärkers auf das Ausgangssignal haben.

\subsection{Methoden}
bla bla Methoden

\subsection{Ergebnisse}
% A)Simulieren Sie für Werte von R1 = R2 = 0Ω ein Bodediagramm der Gesamtschaltung zwischen 100Hz und 10kHz.

% B) Wie erklären Sie sich den Einbruch des Amplitudengangs bei der Grenzfrequenz Ihrer Filterschaltungen?

% C) Fügen Sie zwischen Tiefpass und Addierschaltung einen invertierenden Verstärker mit Verstärkungsfaktor 1 ein und wiederholen Sie die Simulation aus a). Was beobachten Sie nun? Wie nennt sich ein Filter, welches das beobachtete Übertragungsverhalten aufweist?

% D) Simulieren Sie weitere Bodediagramme der Gesamtschaltung für (R1 = 10kΩ, R2 = 90kΩ) und (R1 = 90kΩ, R2 = 10kΩ)

% E)Bauen Sie die Addierschaltung inklusive invertierendem Verstärker jetzt auch auf Ihrem Steckbrett auf. Verwenden Sie für die Widerstände R1 und R2 die beiden Potentiometer (0...100kΩ)desBauteilsortiments.FührenSieBodediagramm-MessungenderGesamtschaltung für R1 = R2 = 0Ω und bei ein paar weiteren Poti-Einstellungen durch.

% F)  Nun sollen zwei Sinussignale verschiedener Frequenz überlagert und auf den Eingang des Equalizers gegeben werden. Nutzen Sie zur Überlagerung die gleiche Schaltung wie in Abbildung 4, jedoch mit Kanal A und Kanal B des DAC an den beiden Enden des Spannungsteilers.ErzeugenSiemitdemSignalgeneratoranKanalAeineFrequenzvon ca. 250Hz. An Kanal B soll ein Sinus mit ca. 5kHz ausgegeben werden. Dies erreichen Sie, indem Sie einen Frequenzmultiplikator für Kanal B von 20 einstellen. Nehmen Sie mit dem Oszilloskop das Eingangssignal und die Ausgangssignale bei den beiden Extremstellungen auf (R1 = 0Ω, R2 = 100kΩ und umgekehrt).


\subsection{Diskussion}
bla bla Diskussion


