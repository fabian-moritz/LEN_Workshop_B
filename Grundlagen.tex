\section{Einleitung}

\subsection{Motivation}
Das Herausfiltern einzelner Frequenzen oder Frequenzbänder spielt in fast jedem alltäglich benutzen elektronischen Geräts eine essentielle Bedeutung. Ein prominentes Beispiel ist das Radio, welches einen möglichst schmalen Frequenzbereich der elektromagnetischen Radiowellen im Bereich von 87,5 MHz bis 108 Mhz \cite{radio} herrausfiltert, verstärkt und als Audiosignal ausgibt.
Auch in analogen Audiomischpulten sind eine Menge Filterschaltungen verbaut. So ermöglichen Filterschaltungen dort verscheidene Frequenzbänder, die auch in ihrer Breite und ihrem Q-Faktor variabel sein können, abzudämpfen oder zu verstärken. Im Bereich der Veranstaltungstechnik werden mittels Bandstop-filtern Raumresonanzfrequenzen herrausgefiltert, um Rückkopplungen zu vermeiden. Trotz des simplen Aufbaus dieses Problems, lässt sich die weitreichende Bedeutung der Filterschaltungen für die Elektrotechnik erkennen.

\subsection{Grundlagen}
blabla technische Grundlagen
